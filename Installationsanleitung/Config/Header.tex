\documentclass[11pt]{scrbook}
\KOMAoptions{draft = false}
\KOMAoptions{paper = a4}
\KOMAoptions{twoside = false}
\KOMAoptions{bibliography=totoc}
\KOMAoptions{listof=totoc}
\KOMAoptions{numbers=noenddot}


\usepackage[%
  backend=bibtex      % biber or bibtex
 ,style=alphabetic    % Alphabeticalsch
 %,style=numeric-comp  % numerical-compressed
 %,sorting=true        % no sorting
 ,sortcites=true      % some other example options ...
 %,block=none
 %,citestyle=none 
 ,indexing=true
 %,citereset=none
 ,isbn=true
 ,url=true
 ,doi=true            % prints doi
 ,natbib=true         % if you need natbib functions
]{biblatex}

\newcommand\tab[1][1cm]{\hspace*{#1}}
\DeclareFieldFormat[online]{title}{#1 [online].}
\DeclareFieldFormat[online]{urldate}{\\Zuletzt gepr\"uft am: \mkbibemph{#1}}
\DeclareFieldFormat[online]{lastupdate}{\\Letzte Aktualisierung: \mkbibemph{#1}}
\DeclareFieldFormat[online]{url}{\\Verf\"ugbar unter: \url{#1}.}

\DeclareFieldFormat[misc]{title}{\emph{#1}.}
\DeclareFieldFormat[misc]{mitarbeiter}{\\Mitarbeiter: #1}
\DeclareFieldFormat[misc]{urldate}{\\Zuletzt gepr\"uft am: #1}
\addbibresource{literatur/literatur} %Bibliographie

%
% Randabstände einstellen
%
\usepackage[top=2.5cm, bottom=3cm, left=2.0cm, right=2.0cm]{geometry}
%
% Das Setspace Paket ermöglicht es auch recht einfach Weise den Zeilenabstand zu ändern.
%
\usepackage{setspace}

%
% Paket für Übersetzungen ins Deutsche
%
\usepackage[ngerman]{babel}
\usepackage{subcaption}
\usepackage{xcolor}
%
% Pakete um Latin1 Zeichnensätze verwenden zu können und die dazu
% passenden Schriften.
%
\usepackage[utf8]{inputenc}
\usepackage[T1]{fontenc}
\definecolor{lightgray}{gray}{0.9}

\newcommand\greybox[1]{%
  \vskip\baselineskip%
  \par\noindent\colorbox{lightgray}{%
    \begin{minipage}{\textwidth}\vspace{6pt}#1\vspace{6pt}\end{minipage}%
  }%
  \vskip\baselineskip%
}
\parindent 24pt

\RequirePackage{colortbl}                        % Farbige Tabellen
\RequirePackage{tabularx}

%%  Darstellung von Formeln (Mathe): %%%%%%%%%%%%%%%%%%%%%%%%%%%%%%%%%%
\RequirePackage{siunitx}        % für m^2 unterm Bruch
\RequirePackage{amsmath}        % Matheumgebung
\RequirePackage{amssymb}        % Symbole
\RequirePackage{cancel}         % Formeln durchstreichen

%%  Service-Pakete, die das Leben einfacher machen  %%%%%%%%%%%%%%%%%%%
\RequirePackage{eurosym}        % EuroSymbol
\RequirePackage{url}            % Schönere URLs
\RequirePackage{lipsum}         % Blindtext, zum Testen
\RequirePackage{xspace}         % intelligente Leerzeichen
\RequirePackage{nameref}        % Mehr Arten der Referenzierung
\RequirePackage{ragged2e}       % Blocksatz ausschalten (z.B. in Tabellen)
\RequirePackage{expdlist}       % Erweiterte discription-Umgebung
\RequirePackage{paralist}       % Schönere Listen

\RequirePackage{calc}           % Interne Berechnungen
\setdefaultitem{\raise0.3ex\hbox{\tiny\(\blacksquare\)}}
               {\raise0.3ex\hbox{\tiny\(\blacksquare\)}}
  						 {\raise0.3ex\hbox{\tiny\(\blacksquare\)}}
               {\raise0.3ex\hbox{\tiny\(\blacksquare\)}}
               


%
% Paket für Quotes
%
\usepackage[babel,french=guillemets,german=swiss]{csquotes}


%
% Paket zum Erweitern der Tabelleneigenschaften
%
\usepackage{array}

%
% Paket für schönere Tabellen
%
\usepackage{booktabs}

\usepackage{float}



%
% Paket um Grafiken einbetten zu können
%
\usepackage{graphicx}
\usepackage{caption}

%
% Spezielle Schrift im Koma-Script setzen.
%
\setkomafont{sectioning}{\normalfont\bfseries}
\setkomafont{captionlabel}{\normalfont\bfseries} 
\setkomafont{pagehead}{\normalfont\bfseries} % Kopfzeilenschrift
\setkomafont{descriptionlabel}{\normalfont\bfseries}

%
% Zeilenumbruch bei Bildbeschreibungen.
%
\setcapindent{0.5em}

%
% Kopf und Fußzeilen
%
\newcommand{\autor}{870723 - 870019} 
\newcommand{\titel}{\trtitleDE} 
\newcommand{\abgabe}{\trdate} 

\usepackage{blindtext}
 
\usepackage[automark, headsepline, footsepline, plainfootsepline]{scrlayer-scrpage}
 
\pagestyle{scrheadings}
\clearscrheadfoot
\automark[section]{chapter}
\ihead[]{\leftmark}
\ohead[]{\rightmark}
\ifoot[\autor]{\autor}
\cfoot[\abgabe]{\abgabe}
\ofoot[Seite\ \pagemark]{Seite\ \pagemark}
%\ofoot[Seite\ \pagemark\ von XXX]{Seite\ \pagemark\ von XXX}
\addtokomafont{pageheadfoot}{\linespread{2}\selectfont} 

%
% mathematische symbole aus dem AMS Paket.
%
\usepackage{amsmath}
\usepackage{amssymb}

%
% Type 1 Fonts für bessere darstellung in PDF verwenden.
%
%\usepackage{mathptmx}           % Times + passende Mathefonts
%\usepackage[scaled=.92]{helvet} % skalierte Helvetica als \sfdefault
\usepackage{lmodern}            % Courier als \ttdefault
\renewcommand*\familydefault{\sfdefault}

%
% Paket um Textteile drehen zu können
%
\usepackage{rotating}

%
% Paket für Farben im PDF
%
\usepackage{color}

%
% Paket um LIstings sauber zu formatieren.
%
\usepackage[savemem]{listings}

% show the filename of files included with \lstinputlisting; also try caption instead of title

%
% ---------------------------------------------------------------------------
%
%
% Hier Werden die capter||section||subsection Überschriften Definiert
%
\renewcommand*\chapterheadstartvskip{\vspace*{-0.3cm}}

\makeatletter
\renewcommand\section{\@startsection{section}{1}{\z@}%
  {-3.5ex \@plus -1ex \@minus -.2ex}%
  {2.3ex \@plus.2ex}%
  {\ifnum \scr@compatibility>\@nameuse{scr@v@2.96}\setlength{\parfillskip}{\z@
      plus 1fil}\fi
    \raggedsection\normalfont\sectfont\size@section}%
}
\renewcommand\subsection{\@startsection{subsection}{2}{\z@}%
  {-3.25ex\@plus -1ex \@minus -.2ex}%
  {1.5ex \@plus .2ex}%
  {\ifnum \scr@compatibility>\@nameuse{scr@v@2.96}\setlength{\parfillskip}{\z@
      plus 1fil}\fi
    \raggedsection\normalfont\sectfont\size@subsection
  }%
}
\makeatother

%
% Neue Umgebungen
%
\newenvironment{ListChanges}%
	{\begin{list}{$\diamondsuit$}{}}%
	{\end{list}}

%
% aller Bilder werden im Unterverzeichnis figures gesucht:
%
\graphicspath{{Bilder/}}

%
% Anführungsstriche mithilfe von \textss{-anzufuehrendes-}
%
\newcommand{\textss}[1]{"`#1"'}

%
% Strukturiertiefe bis subsubsection{} möglich
%
\setcounter{secnumdepth}{3}

%
% Dargestellte Strukturiertiefe im Inhaltsverzeichnis
%
\setcounter{tocdepth}{3}

%
% Zeilenabstand wird um den Faktor 1.5 verändert
%
%\renewcommand{\baselinestretch}{0.5}

%
% Abkürzungsverzeichnis
%
\usepackage[intoc]{nomencl}
% Befehl umbenennen in abk
\let\abk\nomenclature
% Deutsche Überschrift
\renewcommand{\nomname}{Abkürzungsverzeichnis}
% Punkte zw. Abkürzung und Erklärung
\setlength{\nomlabelwidth}{.20\hsize}
\renewcommand{\nomlabel}[1]{#1 \dotfill}
% Zeilenabstände verkleinern
\setlength{\nomitemsep}{-\parsep}
\renewcommand{\nomname}{Abkürzungsverzeichnis}
\makenomenclature


% Tabellen Formatierung
% http://golatex.de/tabellen-schoener-formatieren-t5986.html
\newcolumntype{C}{>{\centering\arraybackslash}X}
\newcommand{\intab}[1]{\begin{tabular}{*{1}{T}} #1\end{tabular}}
\newcommand{\intabb}[2]{\begin{tabular}{*{2}{T}} #1 & #2 \end{tabular}}
\newcommand{\intabbb}[3]{\begin{tabular}{*{3}{T}} #1 & #2 & #3 \end{tabular}}
\newcolumntype{T}{>{\centering\arraybackslash}p{2cm}}

%
% Paket für Links innerhalb des PDF Dokuments
%
\definecolor{LinkColor}{rgb}{255,0,0}
\usepackage[%
	pdftitle={Titel},% Titel der Diplomarbeit
	pdfauthor={Autor},% Autor(en)
	pdfcreator={LaTeX, LaTeX with hyperref and KOMA-Script},% Genutzte Programme
	pdfsubject={Betreff}, % Betreff
	pdfkeywords={Keywords}]{hyperref} % Keywords halt :-)
\hypersetup{pdfborder=LinkColor,% Definition der Links im PDF File
	linkcolor=LinkColor,%
	citecolor=LinkColor,%
	filecolor=LinkColor,%
	menucolor=LinkColor,%
	urlcolor=LinkColor}